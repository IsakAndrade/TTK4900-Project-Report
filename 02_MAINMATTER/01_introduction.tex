\section{Introduction}

\subsection{Background}
This is something to add to our Bibliography. \cite{Swarm_M138_manual}
The \emph{Micromouse project} is to be done in collaboration with the \emph{Norwegian University of Science and Technology (NTNU)}. Aiming to create a set of working robots that 
may be used in future project. There has already been done similar work at NTNU in the \emph{SLAM-robot project} where there are lessons and experience to build upon. The \emph{Micromouse project}
will therefore not be an entirely new endevour, but rather one that keeps building upon already existing work. There are however aspects where this project will deviate from its precursor. Where this is the case
it will be made explicitly clear to the reader.

\subsection{Thesis}

\subsection{Structure of the Report}
This report is mainly intended to lay the foundation for a master project on the upcoming semester. It is more or less a planning stage. Concisting mainly of discussing approaches that should be used. Trade offs
that will be made etc. There is therefore quite a high probability that the structure of this report will deviate from other more typical reports. Especially as there is a need for funding and build expertice.

The uncertainty related to the budget is quite large, and there is a real worry that the university will not be able to provide to much liquidity. Especially as this project is new, and mainly driven by single student
Isak Herrera Sæternes. Onboarding additional members is however a possiblity and may lead to increased funding. This is not that likely, and there will be made an effort to seek monetary support from elsewhere. There
is therefore a need to seek funds elsewhere, and an effort will be made by the project members to apply for external funds as well. Creating a good product requires several design iterations of both hardware and software
these iterations are costly and design trade offs will have to be made.

